\documentclass[12pt, t]{beamer}
\usepackage{amsmath}
\usepackage{setspace}
\usepackage{float} 
\usepackage{multido}
\usepackage{multirow}
\usepackage{array}
\usepackage{enumerate}
\usepackage{booktabs}
\usepackage{indentfirst} 
\usepackage[style=mla]{biblatex}
\usepackage{setspace}
\usepackage{subcaption}
\usepackage{hyperref}
\usepackage{textpos}

\makeatletter
\let\@@magyar@captionfix\relax
\makeatother

\definecolor{Turquoise3}{RGB}{0, 134, 139}
\renewcommand{\emph}[1]{{\color{Turquoise3}\textsl{#1}}}
\newcommand{\C}{\mathbb{C}} \newcommand{\F}{\mathbb{F}} \newcommand{\R}{\mathbb{R}} \newcommand{\Q}{\mathbb{Q}}
\newcommand{\N}{\mathbb{N}}
\newcommand{\myseries}[2]{$#1_1,#1_2,\dots,#1_#2$}
\newcommand{\nullspace}{~\\[15pt]}
\newcommand{\remark}{\textbf{Remark: }}
\newcommand{\scp}[2]{\langle\,#1\,,\,#2\,\rangle} \newcommand{\scpp}{\langle\,\cdot\,,\,\cdot\,\rangle}


\usetheme{Madrid}
\setbeamertemplate{navigation symbols}{}

\addtobeamertemplate{frametitle}{}{
\begin{textblock*}{100mm}(0.85\textwidth,-1cm)
\includegraphics[height=1cm]{logo.png}
\end{textblock*}}

\definecolor{themecolor}{RGB}{25,25,112} 

\usecolortheme[named=themecolor]{structure}

\setbeamertemplate{items}[default]

\hypersetup{
    colorlinks=true,
    linkcolor=themecolor,
    filecolor=themecolor,      
    urlcolor=themecolor,
    citecolor=themecolor,
}

\title{VV285 RC Mid1}
\subtitle{\textbf{Elements of Linear Algebra}\\\large Determinant}
\institute[UM-SJTU JI]{Univerity of Michigan-Shanghai Jiao Tong University Joint Institute}
\author{Xingjian Zhang}

\begin{document}

\begin{frame}
    \titlepage
    \begin{center}
        \includegraphics[height=2cm]{logo2.png}
    \end{center}
\end{frame}

\section{Determinants}
\begin{frame}
    \frametitle{Outline}
    \begin{spacing}{2.5}
        
        \tableofcontents[currentsubsection,hideothersubsections,sectionstyle=hide]        
    \end{spacing}
    
        % \begin{enumerate}
    %     \item Determinant in $\R^2$ and $\R^3$
    %     \item Definition of Determinant
    %     \item Properties of Determinant
    %     \item Practical Way to Calculate Determinant by Hand
    %     \item Find Inverse by Determinant
    %     \item Determinants and System of Equations
    % \end{enumerate}

\end{frame}

\subsection{Determinants in two- and three-dimensional space}
\begin{frame}
    \frametitle{Determinant in $\R^2$ and $\R^3$}
    \begin{itemize}
        \item Determinant in $\R^2$
              \begin{equation}\label{eq27}
                  \det:\text{Mat}(2\times 2;\R)\to\R,\qquad\det\begin{pmatrix}
                      a_1 & b_1 \\
                      a_2 & b_2
                  \end{pmatrix}=a_1b_2-a_2b_1.
              \end{equation}
        \item Determinant in $\R^3$
              \begin{equation}\label{eq28}
                  \det:\R^3\times\R^3\times\R^3\to\R,\qquad\det(a,b,c)=\scp{a\times b}{c}.
              \end{equation}
        \item Cross Product
              \begin{equation}\mathbf{u} \times \mathbf{v}=\left|\begin{array}{ccc}
                      \mathbf{e_1} & \mathbf{e_2} & \mathbf{e_3} \\
                      u_{1}        & u_{2}        & u_{3}        \\
                      v_{1}        & v_{2}        & v_{3}
                  \end{array}\right|\end{equation}
    \end{itemize}
\end{frame}

\subsection{Definition of Determinants}
\begin{frame}
    \frametitle{Formulas of Determinant}
    For every $n\in\N,n>1$, there exists a unique, normed, alternating $n$-multilinear form det:$\underbrace{\R^n\times\cdots\times\R^n}_{n~\text{times}}\cong\text{Mat}(n\times n;\R)\to\R$.\\[7pt]
    Furthermore,\\
    (Using Permutation:  \textit{Leibnitz Formula})
    \begin{equation}\label{1.7.17}
        \det(a_1,\ldots,a_n)=\det A=\sum_{\pi\in S_n}\text{sgn}\,\pi a_{\pi(1)1}\cdots a_{\pi(n)n},
    \end{equation}
    (Using Recursion:  \textit{Laplace Expansion})
    \begin{equation}\label{1.7.21}
        \det A=\sum_{i=1}^{n}(-1)^{i+j}a_{ij}\det A_{ij}
    \end{equation}
\end{frame}

\subsection{Properties of Determinants}
\begin{frame}[allowframebreaks]
    \frametitle{Properties of Determinant}
    Three basis properties:
    \begin{itemize}
        \item n-multilinear
        \item normed
        \item alternating\\
              (when multilinearity holds,)
              \begin{itemize}
                  \item[(i)] $f$ is alternating
                  \item[(ii)] $f(a_1,\ldots,a_{j-1},a_j,a_{j+1},\ldots,a_{k-1},a_k,a_{k+1},\ldots,a_p)$\\ \begin{flushright}
                            $=-f(a_1,\ldots,a_{j-1},a_k,a_{j+1},\ldots,a_{k-1},a_j,a_{k+1},\ldots,a_p)$
                        \end{flushright}
                  \item[(iii)] $f(a_1,\ldots,a_p)=0$ if \myseries{a}{p} are linearly dependent.
              \end{itemize}
    \end{itemize}
    \nullspace
    \remark The property of alternating enables determinant to test whether a matrix's column vectors are linearly independent.
    \newpage
    Properties regarding elementary column operations:
    \begin{itemize}
        \small
        \item The determinant of a matrix $A$ changes sign if two columns of $A$ are interchanged, e.g.,
              \[\det(a_2,a_1,\ldots,a_n)=-\det(a_1,a_2,\ldots,a_n)\]
        \item Multiplying all the entries in a column with a number $\lambda$ leads to the determinant being multiplied by this constant:
              \[\det(a_1,\ldots,\lambda a_j,\ldots,a_n)=\lambda\det(a_1,\ldots,a_j,\ldots,\ldots,a_n)\]
        \item Adding a multiple of a column to another column does not change the value of the determinant:
              \[\det(a_1,\ldots,a_j,\ldots,a_k+\lambda a_j,\ldots,a_n)=\det(a_1,\ldots,a_j,\ldots,a_k,\ldots,a_n)\]
    \end{itemize}
    \remark Properties regarding elementary row operations are analogous.
    \newpage
    Properties regarding matrix operations:
    \begin{itemize}
        \item Matrix Transpose$$\det A=\det A^T$$
        \item Matrix Product $$\det (AB)=\det A \det B$$
        \item Matrix Inverse $$\det A^{-1}=\dfrac{1}{\det A}$$
        \item Calculate Matrix Inverse\[A^{-1}=\frac{1}{\det A}A^{\sharp}\]
    \end{itemize}
    \remark Another way to find inverse is to use the Gau\ss-Jordan Algorithm.
\end{frame}

\subsection{Practical Ways to Calculate Determinants}
\begin{frame}
    \frametitle{Practical Ways to Calculate Determinants}
    Let $A\in\text{Mat}(n\times n)$ have upper triangular form, i.e.,
    \[A=\begin{pmatrix}
            \lambda_1 &        & *         \\
                      & \ddots &           \\
            0         &        & \lambda_n
        \end{pmatrix}\]
    for diagonal elements \myseries{\lambda}{n}$\in\R$ and arbitrary values (denoted by $*$) above the diagonal. Then
    \[\det A=\lambda_1\cdots\lambda_n.\]
    \\[9pt]
    \remark
    This method can be applied to calculate determinants of matrices $A\in\text{Mat}(n\times n)$ when $A$ is first transformed to upper triangular form using elementary matrix manipulations. This is of practical use for $n\geq 4$.


\end{frame}

\subsection{Determinants and Systems of Equations}
\begin{frame}[allowframebreaks,allowdisplaybreaks]
    \frametitle{Determinants and Systems of Equations}
    Determinants have a deep relation with the solutions of systems of equations. For a $n\times n$ matrix $A$, \\[9pt]

    When $\det A \neq 0$, we have

    \begin{enumerate}
        \item $A$ is invertible,
        \item $\ker A = \{0\}$ and $\dim \ker A = 0$,
        \item $\text{ran }A=\R^n$ and $\dim \text{ran }A=n$,
        \item The column vectors $a_{.k}$ and row vectors $a_{j.}$ of $A$ are linear independent.
        \item $\text{rank }A=n$
        \item $Ax=b$ has a unique solution $x=A^{-1}b$ for any $b\in\R^n$.
    \end{enumerate}

    \newpage
    When $\det A = 0$, we have

    \begin{enumerate}
        \item $A$ is not invertible,
        \item $\ker A \neq \{0\}$ and $\dim \ker A > 0$,
        \item $\text{ran }A \subsetneq  \R^n$ and $\dim \text{ran }A<n$,
        \item The column vectors $a_{.k}$ and row vectors $a_{j.}$ of $A$ are linear dependent.
        \item $\text{rank }A<n$
        \item $Ax=0$ has (infinite) non-trivial solutions $x\in \ker A$.
    \end{enumerate}
    \nullspace
    \remark
    These arguments connect everything we learned, from systems of linear equations to determinants. Ask yourself whether you can understand the relationship between each argument.
\end{frame}

\end{document}