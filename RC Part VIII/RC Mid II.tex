\documentclass[10pt, t]{beamer}
\usepackage{amsmath}
\usepackage{setspace}
\usepackage{float} 
\usepackage{multido}
\usepackage{multirow}
\usepackage{array}
\usepackage{enumerate}
\usepackage{booktabs}
\usepackage{indentfirst} 
\usepackage[style=mla]{biblatex}
\usepackage{setspace}
\usepackage{subcaption}
\usepackage{hyperref}
\usepackage{textpos}

\makeatletter
\let\@@magyar@captionfix\relax
\makeatother

\definecolor{Turquoise3}{RGB}{0, 134, 139}
\renewcommand{\emph}[1]{{\color{Turquoise3}\textsl{#1}}}
\newcommand{\C}{\mathbb{C}} \newcommand{\F}{\mathbb{F}} \newcommand{\R}{\mathbb{R}} \newcommand{\Q}{\mathbb{Q}}
\newcommand{\N}{\mathbb{N}}
\newcommand{\myseries}[2]{$#1_1,#1_2,\dots,#1_#2$}
\newcommand{\nullspace}{~\\[15pt]}
\newcommand{\remark}{\textbf{Remark: }}
\newcommand{\scp}[2]{\langle\,#1\,,\,#2\,\rangle} \newcommand{\scpp}{\langle\,\cdot\,,\,\cdot\,\rangle}


\usetheme{Madrid}
\setbeamertemplate{navigation symbols}{}

\addtobeamertemplate{frametitle}{}{
\begin{textblock*}{100mm}(0.85\textwidth,-1cm)
\includegraphics[height=1cm]{logo.png}
\end{textblock*}}

\definecolor{themecolor}{RGB}{25,25,112} 

\usecolortheme[named=themecolor]{structure}

\setbeamertemplate{items}[default]

\hypersetup{
    colorlinks=true,
    linkcolor=themecolor,
    filecolor=themecolor,      
    urlcolor=themecolor,
    citecolor=themecolor,
}

\title{VV285 RC Mid2}
\subtitle{\textbf{Continuity, Differentiability, Integrability}\\\large  Integration in Practice}
\institute[UM-SJTU JI]{Univerity of Michigan-Shanghai Jiao Tong University Joint Institute}
\author{Xingjian Zhang}

\begin{document}

\begin{frame}
    \titlepage
    \begin{center}
        \includegraphics[height=2cm]{logo2.png}
    \end{center}
\end{frame}

\section{Integration in Practice}
\begin{frame}
    \frametitle{Outline}
    \begin{spacing}{2.5}
        \tableofcontents[currentsubsection,hideothersubsections,sectionstyle=hide]
    \end{spacing}
\end{frame}
\subsection{Integration over Jordan-Measurable Sets}

\begin{frame}
    \frametitle{Jordan Measure}
    Let $\Omega\subset\R^n$ be a bounded non-empty set. We define the \emph{outer} and \emph{inner volume} of $\Omega$ by
    \small
    \begin{align*}
        \overline{V}(\Omega)  & :=\inf\left\{\sum_{k=0}^{r}|Q_k|:r\in\N,\,
        Q_0,\ldots,Q_r\in\mathcal{Q}_n,\Omega\subset
        \bigcup\limits_{k=1}^rQ_k\right\},                                 \\
        \underline{V}(\Omega) & :=\sup\left\{\sum_{k=0}^{r}|Q_k|:r\in\N,\,
        Q_0,\ldots,Q_r\in\mathcal{Q}_n,\Omega\supset
        \bigcup\limits_{k=1}^rQ_k,\bigcap\limits_{k=1}^rQ_k
        =\emptyset\right\}.
    \end{align*}
    Let $\Omega\subset\R^n$ be a bounded set. Then $\Omega$ is said to be \emph{(Jordan) measurable} if either
    \begin{enumerate}[(i)]
        \item $\overline{V}(\Omega)=0$ or
        \item $\overline{V}(\Omega)=\underline{V}
                  (\Omega)$.
    \end{enumerate}
    In the first case, we say that $\Omega$ has (Jordan) \emph{measure zero}, in the second case we say that
    \[|\Omega|:=\overline{V}(\Omega)=\underline{V}
        (\Omega)\]
    is the Jordan measure of $\Omega$.

    \textbf{Remark:} Notice that $r\in\N$ is a finite number. On the one hand, for outer volume, we need to find a finite collection of sets ``cover'' the $\Omega$. On the other hand, for inner volume, we also need a finite collection of sets to be ``covered'' by the $\Omega$. Therefore $\Q\cap[0,1]$ is not Jordan measureable.
\end{frame}

\begin{frame}
    \frametitle{Integration over Jordan-Measurable Sets}
    \emph{15.17. Corollary.} Let $\Omega \subset \mathbb{R}^{n}$ be a bounded Jordan-measurable set and let $f: \Omega \rightarrow \mathbb{R}$ be continuous a.e. Then $f$ is integrable on $\Omega$.
    \nullspace
    That's what we learned in the course. However, we had a hidden precondition here: The function $f$ needs to be bounded on its domain. And in this whole section, we consider only bounded functions. So, to be more precise, the statement should be:
    \nullspace
    Let $\Omega \subset \mathbb{R}^{n}$ be a bounded Jordan-measurable set and let $f: \Omega \rightarrow \mathbb{R}$ be (bounded) and (continuous a.e.). Then $f$ is integrable on $\Omega$.
    \nullspace
    \textbf{Remark:} Despite the fact that we only consider bounded functions in this
    section, we should notice that an unbounded function can still have
    integrals. (Improper Riemann Integrals) For example, $\int_0^1\frac{1}{\sqrt{x}}dx$. Be cautious in the exam if you meet an improper integral. Its integral might be bounded, then exists.
\end{frame}

\subsection{Fubini's Theorem}
\begin{frame}
    \frametitle{Fubini's Theorem}
    Let $Q_1$ be an $n_1$-cuboid and $Q_2$ an $n_2$-cuboid so that $Q:=Q_1\times Q_2\subset\R^{n_1+n_2}$ is an $(n_1+n_2)$-cuboid. Assume that $f:Q\to\R$ is integrable on $Q$ and that for every $x\in Q_1$ the integral
    \[g(x)=\int_{Q_2}f(x,\cdot)\]
    exists. Then
    \[\int_Qf=\int_{Q_1\times Q_2}f=\int_{Q_1}g=\int_{Q_1}\left(\int_{Q_2}f\right).\]
    \nullspace
    \textbf{Remark:} This is a very powerful tool. So that we can divide and conquer a integral in $\R^n$. Moreover, if both $\int_{Q_2}f(x,\cdot)$ and $\int_{Q_1}f(\cdot,y)$ exist, we can swap the order of integration. This sometimes simplify our work. See the following examples.
\end{frame}


\begin{frame}
    \frametitle{Exercise}
    Please find the value of following integrals:
    $$\int_{0}^{1} \int_{x}^{1} \sin \left(y^{2}\right) d y d x$$
    $$\int_{0}^{1} \int_{x}^{1} e^{y^{2}} d y d x$$
    \pause
    The primitives of $e^{y^{2}}$ and $\sin \left(y^{2}\right)$ are both non-elementary functions, so it's hard for us to integrate them in the original sequence. So we want to change the order of variables. We verify that they are exchangeable because both integrals $\int_{Q_2}f(x,\cdot)$ and $\int_{Q_1}f(\cdot,y)$ exist (You need to prove this in exam). The region can be express by two inequalities:
    \[
        0 \leq x \leq 1, \quad x \leq y \leq 1
    \]
    which is equivalent to
    \[
        0 \leq y \leq 1, \quad 0 \leq x \leq y
    \]
\end{frame}

\begin{frame}
    \frametitle{Exercise}
    Therefore
    \[
        \begin{aligned}
            \int_{0}^{1} \int_{x}^{1} \sin \left(y^{2}\right) d y d x & =\int_{0}^{1} \int_{0}^{y} \sin \left(y^{2}\right) d x d y                            \\
                                                                      & =\int_{0}^{1} y \sin \left(y^{2}\right) d y                                           \\&=               -\left.\frac{\cos \left(y^{2}\right)}{2}\right|_{0} ^{1}\\&=\frac{1-\cos 1}{2}          \\
            \int_{0}^{1} \int_{x}^{1} e^{y^{2}} d y d x               & =\int_{0}^{1} \int_{0}^{y} e^{y^{2}} d x d y                                          \\
                                                                      & =\int_{0}^{1} y e^{y^{2}} d y=\left.\frac{e^{y^{2}}}{2}\right|_{0} ^{1}=\frac{e-1}{2}
        \end{aligned}
    \]


\end{frame}

\subsection{Substitution Rule}
\begin{frame}
    \frametitle{Substitution Rule}
    Here is another powerful tool named substitution rule. Let $\Omega\subset\R^n$ be open and $g:\Omega\to\R^n$ injective and continuously differentiable. Suppose that $\det J_g(y)\neq0$ for all $y\in\Omega$. Let $K$ be a compact measurable subset of $\Omega$. The $g(K)$ is compact and measurable and if $f:g(K)\to\R$ is integrable, then
    \[\int_{g(K)}f(x)\,dx=\int_Kf(g(y))\cdot
        \vert\det J_g(y)\vert\,dy.\]
    \nullspace
    \textbf{Remark:} Important applications of substitution rule that you should be familiar with:
    \begin{enumerate}
        \item polar coordinate in $\R^2$,
        \item cylindrical coordinate in $\R^3$,
        \item spherical coordinate in $\R^3$,
        \item spherical coordinate in $\R^n$
    \end{enumerate}
    In most cases, we use substitution rule to change to some of these coordinate systems where integral is more convenient. So you should be familiar with this process. We will see some examples given by Jin Haoxiang later.
\end{frame}

% \begin{frame}
%     \frametitle{Substitution Rule}
%     Let $m,n>0$ be integers. If the integral $$\int_{B^n} \frac{1}{\|x\|^{m}} d x$$ in $\mathbb{R}^{n}$ exists, what relationship does $n$ and $m$ have?
%     \nullspace
%     \pause $(n \geq m+1)$

%     With the substitution rule, we can use the spherical coordinates in $\mathbb{R}^{n}$
%     \[
%         \begin{aligned}
%             \int_{B^{n}} \frac{1}{\|x\|^{m}} d x & =\int_{K} \frac{r^{n-1} \sin ^{n-2} \theta_{1} \sin ^{n-3} \theta_{2} \ldots \sin \theta_{n-2}}{r^{m}} d r d \theta_{1} d \theta_{2} \cdots d \theta_{n-1}                 \\
%                                                  & =\left(\int_{0}^{\pi} \sin ^{n-2} \theta_{1} d \theta_{1}\right) \cdots\left(\int_{0}^{2\pi} d \theta_{n-1}\right)\left(\int_{0}^{1} r^{n-m-1} d r\right)
%         \end{aligned}
%     \]
%     We can just consider whether $\int_{0}^{1} r^{n-m-1} d r$ exists.
% \end{frame}

\subsection{Parametrized Surface}
\begin{frame}
    \frametitle{Parametrized Surface}
    A \emph{smooth parametrized m-surface in $\R^n$} is a subset $\mathcal{S}\subset\R^n$ together with a locally bijective, continuously differentiable map (parametrization)
    \[\varphi:\Omega\to\mathcal{S},\qquad\quad
        \Omega\subset\R^m,\]
    such that
   $\text{rank}\,D\varphi|_x=m$
    for almost every $x\in\Omega$. If $m=n-1$, $\mathcal{S}$ is said to be a \emph{parametrized hypersurface}.

    Let $\mathcal{S}\subset\R^n$ be a parametrized $m$-surface with parametrization $\varphi:\Omega\to\mathcal{S}$. Then
    \[t_k(p)=\frac{\partial}{\partial x_k}\left.\begin{pmatrix}
            \varphi_1(x) \\
            \vdots       \\
            \varphi_n(x)
        \end{pmatrix}\right|_{x=\varphi^{-1}(p)},\qquad\quad
        k=1,\ldots,m.\]
    is called the \emph{$k$th tangent vector of $\mathcal{S}$ at $p\in\mathcal{S}$} and
    \[T_p\mathcal{S}:=\text{ran}\,D\varphi|_x
        =\text{span}\{t_1(p),\ldots,t_m(p)\}\]
    is called the \emph{tangent space} to $\mathcal{S}$ at $p$.\\
    \textbf{Remark:} Notice that a tangent space is a subspace of $\R^n$. So it contains $0$. 
\end{frame}

\begin{frame}
    \frametitle{Normal Vector to Hypersurface}
    Let $\mathcal{S}\subset\R^n$ be a hypersurface. Then a unit vector that is orthogonal to all tangent vectors to $\mathcal{S}$ at $p$ is called a \emph{unit normal vector to $\mathcal{S}$ at $p$} and denoted by $N(p)$. The vector field
    \[N:\mathcal{S}\to\R^n,\qquad\qquad\quad
        p\mapsto N(p)\]
    is called the \emph{normal vector field} on $\mathcal{S}$.

    \begin{enumerate}[(i)]
        \item A hypersurface that is the boundary of a measurable set $\Omega\subset\R^n$ with non-zero measure is said to be a \emph{closed surface}.
        \item A closed hypersurface is said to have \emph{positive orientation} if the
              normal vector field is chosen so that the normal vectors point
              outwards from $\Omega$. 
    \end{enumerate}
\end{frame}

\begin{frame}
    \frametitle{Area of Hypersurface}

    We define the \emph{scalar surface element of a hypersurface in $\R^3$} by
    \[dA=|\det(t_1,t_2,N)\circ\varphi|dx_1\,dx_2.\]
    Of course, we can generalize this to hypersurfaces in $\R^n$, setting
    \[dA=|\det(t_1,t_2,\ldots,t_{n-1},N)\circ
        \varphi|\,dx_1\,dx_2\ldots dx_{n-1}.\]

    We can then define the area/volume of a hypersurface:
    The \emph{volume} or \emph{area} of $\mathcal{S}$ is defined as
    \[|\mathcal{S}|:=\int_\Omega|\det(t_1,\ldots,t_{n-1},N)
        \circ\varphi(x)|\,dx_1\,dx_2\ldots dx_{n-1}.\]

    \textbf{Remark:} With the help of \emph{metric tensor}, we can then calculate $dA$ without recourse to a normal vector $N$.
\end{frame}

\begin{frame}
    \frametitle{Metric Tensor}
    Let $\mathcal{S}\subset\R^n$ be an $m$-surface with parametrization $\varphi$ and tangent vector fields \myseries{t}{m}. Then $G\in\text{Mat}(m\times m;\R)$ given by
    \[G:=\begin{pmatrix}
            \scp{t_1}{t_1} & \cdots & \scp{t_1}{t_m} \\
            \vdots         & \ddots & \vdots         \\
            \scp{t_m}{t_1} & \cdots & \scp{t_m}{t_m}
        \end{pmatrix}\]
    is said to be the \emph{metric tensor} on $\mathcal{S}$ with respect to $\varphi$.
    \nullspace
    \textbf{Remark:} By the definition of metric tensor, we immediately have
    $$\left|\operatorname{det}\left(t_{1}, \ldots, t_{n-1}, N\right)\right|=\sqrt{\operatorname{det} G}$$
    The significance of metric tensor is: we don't bother ourselves to calculate the normal vector if we want to find $dA$.

\end{frame}

\begin{frame}
    \frametitle{Redefinition Using Metric Tensor}
    We can then equivalently define some concepts using metric tensor.

    \begin{enumerate}
        \item The scalar surface element in $\R^3$:
              \begin{itemize}
                  \item \[dA=|\det(t_1,t_2,N)\circ\varphi|dx_1\,dx_2.\]
                  \item $$d A=\left\|t_{1} \times t_{2}\right\| \circ \varphi(x) d x_{1} d x_{2}$$
              \end{itemize}
        \item The scalar surface element in $\R^n$:
              \begin{itemize}
                  \item \[dA=|\det(t_1,t_2,\ldots,t_{n-1},N)\circ
                            \varphi|\,dx_1\,dx_2\ldots dx_{n-1}.\]
                  \item     \[dA:=\sqrt{g(x)}\,dx\] where $g(x)$ is the determinant of metric tensor $G$ at $x$.
              \end{itemize}
        \item The volume or area of a hypersurface $\mathcal{S}$ in $\R^n$:
              \begin{itemize}
                  \item \[|\mathcal{S}|:=\int_\Omega|\det(t_1,\ldots,t_{n-1},N)
                            \circ\varphi(x)|\,dx_1\,dx_2\ldots dx_{n-1}.\]
                  \item    \[|\mathcal{S}|:=\int_\Omega\sqrt{g(x)}dx \quad\text{(Can be used in non-hypersurface case, more general)}\]
              \end{itemize}
    \end{enumerate}
\end{frame}

\begin{frame}
    \frametitle{Scalar Surface Integral}
    Let $f:\mathcal{S}\to\R$ be a potential function. Then the \emph{(scalar) surface integral of $f$ over $\mathcal{S}$} is defined as
    \[\int_\mathcal{S}f\,dA:=\int_\Omega f\circ\varphi(x)\sqrt{g(x)}\,dx\]
    Intuitively, we can consider scalar surface integral as the total mass of a surface that has non-uniform distribution of density (Recall the scalar line integral where we consider it as the total mass of a non-uniform string).
    \nullspace
    In practice, the calculus is tedious. Here are the general steps of solving a scalar surface integral problem.
    \begin{enumerate}
        \item Clarify the parametrization of the surface. (You might choose convenient coordinates in this step.)
        \item Find the tangent vectors of the surface.
        \item Use the metric tensor or normal vector to find $dA$.
        \item Perform the scalar surface integral.
    \end{enumerate}
    You can directly give the volume/area of some regular objects in the exam without rigorous proof, e.g. sphere, cube, cylinder.

\end{frame}

\begin{frame}
    \frametitle{Summary}
    Make sure you prepare well before the exam. i.e. You know how to perform any of these integrals \textbf{in details}. I did not prepare enough examples for all kinds of integrals we learned due to time limitation. But to better prepared yourself for the exam (and future study), I recommend you to find some exercises that are ugly and tedious enough and solve them by yourself. (Ugly and tedious: There is no symmetry/The region is not a cuboid/The result is not 0/You may need to use substitution rule and Fubini's theorem.) Moreover, do not limit yourself in merely reviewing the integral part. Knowledge like derivatives are very important! And they are easy to be neglected.
    \nullspace
    The grading of Mid2 will be very strict because it is an open-book exam and Mathematica is allowed. To get a full mark of a question (especially regarding integral), make sure you write down all necessary steps that can show your own work rather than simply leave an answer given by Mathematica.
    \nullspace
    Have fun and learn well!
\end{frame}

\end{document}